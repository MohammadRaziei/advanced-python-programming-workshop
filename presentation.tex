%%%%%%%%%%%%%%%%%%%%%%%%%%%%%%%%%%%%%%%%%
% Beamer Presentation
% LaTeX Template
% Version 1.0 (10/11/12)
%
% This template has been downloaded from:
% http://www.LaTeXTemplates.com
%
% License:
% CC BY-NC-SA 3.0 (http://creativecommons.org/licenses/by-nc-sa/3.0/)
%
%%%%%%%%%%%%%%%%%%%%%%%%%%%%%%%%%%%%%%%%%

%----------------------------------------------------------------------------------------
%	PACKAGES AND THEMES
%----------------------------------------------------------------------------------------

\documentclass[10pt]{beamer}


\mode<presentation> {
	
	% The Beamer class comes with a number of default slide themes
	% which change the colors and layouts of slides. Below this is a list
	% of all the themes, uncomment each in turn to see what they look like.
	
	%\usetheme{default}
	%\usetheme{AnnArbor}
	%\usetheme{Antibes}
	%\usetheme{Bergen}
	\usetheme{Berkeley}
	%\usetheme{Berlin}
	%\usetheme{Boadilla}
	%\usetheme{CambridgeUS}
	%\usetheme{Copenhagen}
	%\usetheme{Darmstadt}
	%\usetheme{Dresden}
	%\usetheme{Frankfurt}
	%\usetheme{Goettingen}
	%\usetheme{Hannover}
	%\usetheme{Ilmenau}
	%\usetheme{JuanLesPins}
	%\usetheme{Luebeck}
	%\usetheme{Madrid}
	%\usetheme{Malmoe}
	%\usetheme{Marburg}
	%\usetheme{Montpellier}
	%\usetheme{PaloAlto}
	%\usetheme{Pittsburgh}
	%\usetheme{Rochester}
	%\usetheme{Singapore}
	%\usetheme{Szeged}
	%\usetheme{Warsaw}
	
	% As well as themes, the Beamer class has a number of color themes
	% for any slide theme. Uncomment each of these in turn to see how it
	% changes the colors of your current slide theme.
	
	%\usecolortheme{albatross}
	%\usecolortheme{beaver}
	%\usecolortheme{beetle}
	%\usecolortheme{crane}
	%\usecolortheme{dolphin}
	%\usecolortheme{dove}
	%\usecolortheme{fly}
	%\usecolortheme{lily}
	%\usecolortheme{orchid}
	%\usecolortheme{rose}
	%\usecolortheme{seagull}
	\usecolortheme{seahorse}
	%\usecolortheme{whale}
	%\usecolortheme{wolverine}
	
	%\setbeamertemplate{footline} % To remove the footer line in all slides uncomment this line
	%\setbeamertemplate{footline}[page number] % To replace the footer line in all slides with a simple slide count uncomment this line
	
	%\setbeamertemplate{navigation symbols}{} % To remove the navigation symbols from the bottom of all slides uncomment this line
}

\usepackage{amsmath,amssymb}
\usepackage{bm}
\usepackage{copyrightbox}
\usepackage{listings}
\usepackage{array}
\usepackage{tikz}
\usepackage{adjustbox}
\usepackage{ragged2e}
\usepackage{etoolbox}
\usepackage{subfigure}
\usepackage{colortbl,color,xcolor}


\usepackage{graphicx} % Allows including images
\usepackage{booktabs} % Allows the use of \toprule, \midrule and \bottomrule in tables
\usepackage{fontawesome5}
\usepackage{url}
\usepackage{hyperref}

\usepackage{bibentry}
%\usepackage[backend=biber,style=ieee, citestyle=authoryear]{biblatex}
%\usepackage[style=ieee-alphabetic, backend=bibtex]{biblatex}
\usepackage[style=numeric, backend=bibtex]{biblatex}



%\bibliography{bibfile.bib}
\addbibresource{assets/bibfile.bib}



\makeatletter
\newcommand*{\mkblankfootnote}[1]{%
	\begingroup
	\renewcommand\thefootnote{}%
	\footnotetext{\bibfootnotewrapper{#1}}%
	\endgroup
}

\newcommand*{\mkbibsupercite}[1]{%
	\def\cbx@savedcites{\cbx@footfullcite}%
	\mkbibbrackets{#1}%
	\ifx\cbx@savedkeys\@empty
	\else
	\cbx@savedcites
	\fi}

\DeclareCiteCommand{\supercite}[\mkbibsupercite]
{\gdef\cbx@savedkeys{}}
{\usebibmacro{citeindex}%
	\usebibmacro{cite}%
	{}%
	\xappto\cbx@savedkeys{\thefield{entrykey},}}
{\supercitedelim}
{\protected@xappto\cbx@savedcites{%
		[\thefield{prenote}][\thefield{postnote}]{\cbx@savedkeys}}}

\DeclareCiteCommand{\cbx@footfullcite}
{}
{\mkblankfootnote{%
		\printtext[labelalphawidth]{%
			\usebibmacro{cite}%
		}%
		\setunit{\addspace}%
		\usedriver
		{\DeclareNameAlias{sortname}{default}}
		{\thefield{entrytype}}}}
{}
{}
\makeatother

\renewcommand*{\nameyeardelim}{\addcomma\addspace}

\apptocmd{\frame}{}{\justifying}{} % Allow optional arguments after frame.

\let\olditem\item
\renewcommand\item{\olditem\justifying}

\addtobeamertemplate{navigation symbols}{}{%
	\usebeamerfont{footline}%
	\usebeamercolor[fg]{footline}%
	\hspace{1em}%
	\insertframenumber/\inserttotalframenumber
}


\makeatletter
\@addtoreset{subfigure}{figure}
\makeatother



\makeatletter
\let\old@lstKV@SwitchCases\lstKV@SwitchCases
\def\lstKV@SwitchCases#1#2#3{}
\makeatother
\usepackage{lstlinebgrd}
\makeatletter
\let\lstKV@SwitchCases\old@lstKV@SwitchCases

\lst@Key{numbers}{none}{%
	\def\lst@PlaceNumber{\lst@linebgrd}%
	\lstKV@SwitchCases{#1}%
	{none:\\%
		left:\def\lst@PlaceNumber{\llap{\normalfont
				\lst@numberstyle{\thelstnumber}\kern\lst@numbersep}\lst@linebgrd}\\%
		right:\def\lst@PlaceNumber{\rlap{\normalfont
				\kern\linewidth \kern\lst@numbersep
				\lst@numberstyle{\thelstnumber}}\lst@linebgrd}%
	}{\PackageError{Listings}{Numbers #1 unknown}\@ehc}}
\makeatother
\newcounter{subListing}[subfigure]

\definecolor{codegreen}{rgb}{0,0.6,0}
\definecolor{codegray}{rgb}{0.5,0.5,0.5}
\definecolor{codepurple}{rgb}{0.58,0,0.82}
\definecolor{mygreen}{RGB}{28,172,0} 
\definecolor{mylilas}{RGB}{170,55,241}
\definecolor{backcolour}{rgb}{1,1,0.98}

\lstset{language=Python,%
	backgroundcolor=\color{backcolour},   
	commentstyle=\color{codegreen},
	keywordstyle=\color{blue},
	numberstyle=\tiny\color{codegray},
	stringstyle=\color{codepurple},
	basicstyle=\tt\scriptsize,
	frame = LBtr,
	%frameround=T,
	rulecolor=\color{gray},
	showstringspaces=false,
	numbers=left,%
	numberstyle={\tiny\color{gray}},
	numbersep=8pt,
	breaklines=true,
	%postbreak=\mbox{\textcolor{yellow}{$\hookrightarrow$}\space},
	tabsize=2,
	escapechar=`,
	xleftmargin=1.8 em, 
	framexleftmargin=2em,
}


%----------------------------------------------------------------------------------------
%	TITLE PAGE
%----------------------------------------------------------------------------------------

\definecolor{AtherosPrimary}{rgb}{0.992156,0.380392,0.34509803921}
\definecolor{AtherosSecondary}{rgb}{0.019,0,0.1568} % UBC Blue (primary)
\definecolor{AtherosTertiary}{rgb}{0.215,0.0625,0.515625}

\setbeamercolor{palette primary}{bg=AtherosSecondary,fg=white}
\setbeamercolor{palette tertiary}{bg=AtherosPrimary,fg=white}
\setbeamercolor{palette secondary}{bg=AtherosTertiary,fg=white}

\title{Advanced Python Programming Workshop} % The short title appears at the bottom of every slide, the full title is only on the title page

\logo{\includegraphics[width=1.2cm]{assets/logo}}


\author{Mohammad Raziei} % Your name
\institute[Sharif University] % Your institution as it will appear on the bottom of every slide, may be shorthand to save space
{
	\faUniregistry\ Sharif University of Technology\\ % Your institution for the title page
	\medskip
	\href{mailto://mohammadraziei1375@gmail.com}{\faMailBulk\  mohammadraziei1375@gmail.com}\\ % Your email address
	\medskip
	\href{https://github.com/mohammadraziei}{\faGithub\ mohammadraziei} % Your email address
}
\date{\today} % Date, can be changed to a custom date

\graphicspath {{assets/}}

\newcommand{\mypause}{\pause}

\begin{document}
	
	
	% Slide 1: Title Slide
	\begin{frame}
		\titlepage
	\end{frame}
	
	
	\begin{frame}[allowframebreaks]
		\tableofcontents
	\end{frame}
	
	\section{Introduction to Python}
	\subsection{Advantages and Disadvantages of Python}
	\begin{frame}{Advantages and Disadvantages of Python}
		\textbf{Advantages} 
		\hfill \raisebox{-.5cm}{\includegraphics[width=1cm]{like}} \\[0.1cm]
		\begin{itemize}
			\item Easy to learn and use
			\item Extensive libraries for data analysis (\texttt{Pandas}, \texttt{NumPy}, \texttt{Matplotlib})
			\item Support for machine learning and AI frameworks (\texttt{scikit-learn}, \texttt{TensorFlow}, \texttt{PyTorch})
			\item High-level data manipulation capabilities
			\item Dynamically typed: No need to declare variable types explicitly
			\item Cross-platform compatibility
			\item Versatility for scripting, web development, data visualization, and more
			\item Used for developing almost everything: desktop apps, APIs, web apps, data tools, and more
		\end{itemize}
	\end{frame}
	
	\begin{frame}{Advantages and Disadvantages of Python}
		\textbf{Disadvantages}
		\hfill \raisebox{-.5cm}{\includegraphics[width=1cm]{dislike}} \\[0.1cm]
		\begin{itemize}
			\item Slow execution speed compared to compiled languages (e.g., \texttt{C}, \texttt{C++})
			\item High memory consumption, not ideal for memory-intensive tasks
			\item Limited performance for mobile and embedded systems
			\item Dynamically typed nature can lead to runtime errors
			\item Global Interpreter Lock (\texttt{GIL}) restricts multi-threading
			\item Not the best choice for low-level programming
			\item Slower database access compared to other languages
		\end{itemize}
	\end{frame}
	
	\begin{frame}{Performance Analysis of Programming Languages}
		\begin{minipage}{0.5\linewidth}
			\begin{figure}
				\centering
				\includegraphics[width=\linewidth]{ranking}
				\caption{Programming Language Benchmarks Visualization \cite{actions2024GoodManWEN}}
			\end{figure}
		\end{minipage}
		\mypause
		\begin{minipage}{0.5\linewidth}
			\begin{itemize}
				\item Interpreted Language
				\item Global Interpreter Lock (GIL)
				\item Dynamic Typing
				\item Garbage Collection
			\end{itemize}
		\end{minipage}
	\end{frame}
	
	\subsection{Garbage Collector in Python}
	\begin{frame}{Garbage Collector in Python}
		\begin{itemize}
			\item \textbf{Garbage Collector (GC):} 
			A memory management tool that automatically deallocates unused objects in memory, preventing memory leaks.
			\item \textbf{How it works:}
			\begin{itemize}
				\item Python uses reference counting to keep track of objects.
				\item The GC handles cyclic references (e.g., objects referencing each other).
			\end{itemize}
			\item \textbf{\texttt{gc} module:}
			\begin{itemize}
				\item Provides an interface to the garbage collector.
				\item Functions:
				\begin{itemize}
					\item \texttt{gc.collect()}: Force a garbage collection.
					\item \texttt{gc.get\_stats()}: Get statistics on memory management.
					\item \texttt{gc.disable()}/\texttt{gc.enable()}: Turn GC off/on.
					\item \texttt{gc.garbage}: List of uncollectable objects.
				\end{itemize}
				\item Useful for debugging memory issues in Python applications.
			\end{itemize}
		\end{itemize}
	\end{frame}
	
	\subsection{Popular IDEs}
	\begin{frame}{Comparison of Popular IDEs}
		\begin{adjustbox}{scale=.9}
		\begin{minipage}{1.11\linewidth}
		\begin{itemize}	
			\item \textbf{PyCharm:}
			\hfill\raisebox{-.2cm}{\includegraphics[height=.7cm]{pycharm}}
			\begin{itemize}
				\item Best for professional Python development.
				\item Advanced debugging, intelligent code completion, and project management tools.
			\end{itemize}
			\item \textbf{VS Code:}
			\hfill\raisebox{-.2cm}{\includegraphics[height=.7cm]{vscode}}
			\begin{itemize}
				\item Lightweight and highly customizable.
				\item Extensive extensions for Python, Git integration, and multiple language support.
				\item Built-in support for SSH for remote development.
			\end{itemize}
			\item \textbf{Spyder:}
			\hfill\raisebox{-.2cm}{\includegraphics[height=.7cm]{spyder}}
			\begin{itemize}
				\item Designed for scientific computing and data analysis.
				\item Includes a variable viewer, similar to MATLAB's workspace.
				\item Seamlessly integrates with \texttt{NumPy}, \texttt{Pandas}, and \texttt{Matplotlib}.
			\end{itemize}
			\item \textbf{Jupyter Notebook:}
			\hfill\raisebox{-.2cm}{\includegraphics[height=.7cm]{jupyter}}
			\begin{itemize}
				\item Ideal for interactive coding, data visualization, and educational purposes.
				\item Supports inline plots and markdown for creating documentation-rich notebooks.
			\end{itemize}
		\end{itemize}
		\end{minipage}
		\end{adjustbox}
	\end{frame}
	
	
	\section{Popular Python Libraries}
	\subsection{Scientific Computing}
	\begin{frame}{Popular Python Libraries for Scientific Computing}
		\textbf{Overview of Key Libraries}
		\begin{itemize}
			\item \textbf{NumPy:}
			\begin{itemize}
				\item Provides support for multi-dimensional arrays and matrices.
				\item Includes mathematical functions for efficient numerical computations.
				\item Serves as the foundation for many other libraries like SciPy and Pandas.
			\end{itemize}
			\item \textbf{SciPy:}
			\begin{itemize}
				\item Builds on NumPy to include advanced mathematical, scientific, and engineering functions.
				\item Offers modules for optimization, signal processing, linear algebra, and more.
			\end{itemize}
			\item \textbf{Pandas:}
			\begin{itemize}
				\item Focuses on data manipulation and analysis.
				\item Provides powerful tools for working with structured data, such as DataFrames.
				\item Ideal for handling missing data, merging datasets, and time-series analysis.
			\end{itemize}
		\end{itemize}
		

	\end{frame}
	
	
	\begin{frame}{Popular Python Libraries for Scientific Computing}
			\textbf{Why These Libraries Matter}
	\begin{itemize}
		\item Enable high-level, MATLAB-like functionality for numerical computations and data analysis.
		\item Combine performance with simplicity, making Python a powerful tool for scientific and engineering tasks.
		\item Offer extensive functionality while maintaining flexibility for a wide range of applications.
	\end{itemize}
		\end{frame}
		
	\subsection{Data Visualization}
	\begin{frame}{Popular Python Libraries for Data Visualization}
		\textbf{Overview of Key Libraries}
		\begin{itemize}
			\item \textbf{Matplotlib:}
			\begin{itemize}
				\item A versatile library for creating static, animated, and interactive visualizations.
				\item Supports a wide range of charts, including line plots, bar charts, scatter plots, histograms, pie charts, and more \supercite{matplotlibExamplesx2014}.
				\item Provides a MATLAB-like interface for ease of use.
			\end{itemize}
			\item \textbf{Seaborn:}
			\begin{itemize}
				\item Built on top of Matplotlib, focusing on statistical data visualization.
				\item Simplifies the creation of aesthetically pleasing and informative visualizations.
				\item Includes advanced plot types like heatmaps, pair plots, violin plots, and swarm plots. \supercite{pydataExampleGallery}
			\end{itemize}
		\end{itemize}
	\end{frame}
	
	\subsection{Machine Learning}
	\begin{frame}{Popular ML and DL Libraries}
		\textbf{Overview:}
		\begin{itemize}
			\item \textbf{Keras:} High-level API for deep learning, built on TensorFlow, easy for prototyping.
			\item \textbf{PyTorch:} Dynamic computation graph, widely used in research and production.
			\item \textbf{TensorFlow:} End-to-end platform for machine learning, supports distributed training.
			\item \textbf{Scikit-Learn:} Classical ML library for regression, classification, and clustering.
		\end{itemize}
		
		\textbf{Applications:}
		\begin{itemize}
			\item \textbf{Keras, PyTorch, TensorFlow:} Deep learning for images, NLP, and time-series.
			\item \textbf{Scikit-Learn:} Preprocessing, classical ML, and model evaluation.
		\end{itemize}
	\end{frame}
	
	
	
	\begin{frame}{Web Page Parsing and Applications}
		\textbf{What is Web Parsing?}
		\begin{itemize}
			\item Extracting structured information from web pages.
			\item Used in web scraping and building web crawlers.
		\end{itemize}
		
		\textbf{Applications:}
		\begin{itemize}
			\item Data extraction (e.g., prices, reviews, articles).
			\item Automating data collection for analysis.
			\item Indexing web pages for search engines.
		\end{itemize}
		
		\textbf{Popular Parsers:}
		\begin{itemize}
			\item {html.parser}, {html5lib}, {lxml}, {parsel}, {BeautifulSoup}, {selectolax}.
		\end{itemize}
	\end{frame}
	
	
	\begin{frame}{Comparison of Python HTML Parsers}
		\begin{figure}
			\centering
			\includegraphics[width=.8\textwidth]{parser-benchmark}
			\caption{Performance comparison of Python HTML parsers.}
			\label{fig:parser-performance}
		\end{figure}
	\end{frame}
	
	\begin{frame}{Comparison of Python HTML Parsers}
		\begin{table}
			\centering
			\begin{adjustbox}{scale=.65}
			\begin{tabular}[]{>{\centering\arraybackslash}p{3cm} |*{5}{>{\centering\arraybackslash}p{1.9cm}}}
				\hline
				\rowcolor[gray]{.85}
				Parser & Speed & Memory & Simplicity & {XPath} & {CSS} \\
				\hline
				\cellcolor{white} {html.parser} & \cellcolor{red!50} Low & \cellcolor{green!50} Low & \cellcolor{green!50} High & \cellcolor{red!50} No & \cellcolor{red!50} No \\
				
				\cellcolor{white} {html5lib} & \cellcolor{red!50} Low & \cellcolor{red!50} High & \cellcolor{green!50} High & \cellcolor{red!50} No & \cellcolor{red!50} No \\
				
				\cellcolor{pink!40} {lxml} & \cellcolor{green!50} High & \cellcolor{yellow!50} Medium & \cellcolor{yellow!50} Medium & \cellcolor{green!50} Yes & \cellcolor{green!50} Yes \\
				
				\cellcolor{white} {parsel} & \cellcolor{yellow!50} Medium & \cellcolor{yellow!50} Medium & \cellcolor{green!50} High & \cellcolor{green!50} Yes & \cellcolor{green!50} Yes \\
				
				\cellcolor{white} {BeautifulSoup} & \cellcolor{white} Variable & \cellcolor{white} Variable & \cellcolor{green!50} High & \cellcolor{red!50} No & \cellcolor{red!50} No \\
				
				\cellcolor{pink!40} {selectolax} & \cellcolor{green!50} High & \cellcolor{green!50} Low & \cellcolor{yellow!50} Medium & \cellcolor{red!50} No & \cellcolor{green!50} Yes \\
				\hline
			\end{tabular}
			\end{adjustbox}
			\caption{Comparison of Python HTML Parsers}
			\label{tab:parser-benchmarks}
		\end{table}
	\end{frame}
	
	
	\begin{frame}{Introduction to \texttt{librosa}}
		\textbf{What is \texttt{librosa}?}
		\begin{itemize}
			\item \texttt{librosa} is a Python library for audio and music analysis.
			\item Provides tools for loading, analyzing, and transforming audio data.
			\item Widely used in audio processing and machine learning applications.
		\end{itemize}
		
		\textbf{Key Features:}
		\begin{itemize}
			\item Audio loading and saving in multiple formats.
			\item Feature extraction, such as MFCCs, chroma, and spectral features.
			\item Time-domain and frequency-domain transformations (e.g., STFT, CQT).
			\item Utilities for beat tracking, tempo estimation, and pitch detection.
		\end{itemize}
		
		\textbf{Applications:}
		\begin{itemize}
			\item Speech emotion recognition and audio classification.
			\item Music recommendation systems and beat tracking.
			\item Data augmentation and preprocessing in deep learning models.
		\end{itemize}
	\end{frame}
	
		
		

		
		
		
		\section{Python Annotations}
		
		
	
	\begin{frame}[fragile]{Understanding Python Annotations}
		\textbf{What are Annotations?}
		\begin{itemize}
			\item Annotations are metadata added to function arguments and return values.
			\item They are used for documentation, type hinting, and static code analysis.
			\item Introduced in Python 3.0 and expanded in Python 3.5 with the \texttt{typing} module.
		\end{itemize}
		
%\begin{adjustbox}{scale=.9}
%	\begin{minipage}{1.11\linewidth}
		\begin{lstlisting}[caption=Using Annotations in Python]

def greet(name: str, age: int) -> str:
	"""
	Function to generate a greeting message.
	Args:
		name (str): The name of the person.
		age (int): The age of the person.
	Returns:
		str: A personalized greeting message.
	"""
	return f"Hello {name}, you are {age} years old!"
		\end{lstlisting}
%	\end{minipage}
%\end{adjustbox}
\end{frame}
	
	
	
	
	\begin{frame}{Using \texttt{typing} for Class Annotations}
		\textbf{Example: Annotating a Class with \texttt{typing}}
		\lstinputlisting[caption=Class Annotations with \texttt{typing}]{codes/examples/typing-example.py}
	
		
%		\textbf{Explanation:}
%		\begin{itemize}
%			\item \texttt{name: str} and \texttt{age: int} are basic type hints.
%			\item \texttt{skills: List[str]} specifies a list of strings.
%			\item \texttt{details: Optional[Dict[str, Union[str, int]]]} allows a dictionary with string keys and mixed string/integer values or \texttt{None}.
%			\item Annotations make the code easier to understand and suitable for static analysis.
%		\end{itemize}
	\end{frame}
	
	
	
	\begin{frame}[fragile]{Using \texttt{typeguard} for Runtime Type Checking}
		\begin{lstlisting}[language=Python, caption=Without \texttt{typechecked}]
def func(value: int) -> int:
	""" Multiplies the input value by 2. """
	return value * 2
# Example 1: Correct usage
print(func(5))  # Output: 10
# Example 2: Incorrect usage
print(func("5"))  # No error, Output: "55" (string concatenation)
		\end{lstlisting}
		
		\begin{lstlisting}[language=Python, caption=With \texttt{typechecked}]
from typeguard import typechecked

@typechecked
def func(value: int) -> int:
	""" Multiplies the input value by 2. """
	return value * 2
# Example 1: Correct usage
print(func(5))  # Output: 10
# Example 2: Incorrect usage
print(func("5"))  # TypeError: type of argument "value" must be int; got str instead
		\end{lstlisting}
	\end{frame}
	
	
	\begin{frame}{Introduction to \texttt{typeguard}}
		\textbf{Why Use \texttt{typeguard}\supercite{typeguardUserGuide}?}
		\begin{itemize}
			\item Ensures adherence to type annotations in dynamic typing environments.
			\item Useful for debugging and catching runtime type mismatches.
			\item Adds robustness to Python codebases, especially in large projects.
		\end{itemize}
		\mypause
		\textbf{Key Feature: \texttt{typechecked} Decorator \textcolor{red}{(?)}}
		\begin{itemize}
			\item The \texttt{@typechecked} decorator enforces type annotations on functions and methods.
			\item Validates:
			\begin{itemize}
				\item Function arguments.
				\item Return values.
			\end{itemize}
			\item Raises a \texttt{TypeError} if a mismatch is detected.
		\end{itemize}
	\end{frame}
	
	
	\begin{frame}{Understanding Decorators in Python}
		\textbf{What is a Decorator?}
		\begin{itemize}
			\item A decorator is a function that modifies the behavior of another function or method.
			\item They are used to extend or enhance functionality without modifying the original function's code.
			\item Commonly applied with the \texttt{@decorator\_name} syntax.
		\end{itemize}
		
		\textbf{How Does it Work?}
		\begin{itemize}
			\item A decorator takes a function as input and returns a new function.
			\item The new function wraps the original function, adding additional behavior.
		\end{itemize}
		
		\textbf{Why Use Decorators?}
		\begin{itemize}
			\item Code reusability and modularity.
			\item Useful for logging, validation, access control, etc.
			\item Simplifies and cleans up code by separating concerns.
		\end{itemize}
	\end{frame}
	
	
	\begin{frame}[fragile]{Example: Simple Decorator}
	\lstinputlisting[caption=Basic Decorator Example, lastline=12]{codes/examples/decorator-example.py}
	
	\mypause
	
	\lstinputlisting[firstline=18, lastline=22, firstnumber=13]{codes/examples/decorator-example.py}
	
	\mypause
	
	\lstinputlisting[firstline=13, lastline=17, frame=single,numbers=none]{codes/examples/decorator-example.py}
		
	\end{frame}
	
	\begin{frame}{\texttt{classmethod} and \texttt{staticmethod} as Decorators}
		\textbf{What are \texttt{classmethod} and \texttt{staticmethod}?}
		\begin{itemize}
			\item Both are \textbf{decorators} used to modify the behavior of methods in a class.
			\item They define how methods interact with the class or its instances:
			\begin{itemize}
				\item \texttt{classmethod:} Accesses and modifies class-level data.
				\item \texttt{staticmethod:} Behaves like a regular function but is logically grouped within the class.
			\end{itemize}
		\end{itemize}
	\end{frame}
		
	\begin{frame}[fragile]{Example Code:}
		\lstinputlisting[language=Python, caption=Using \texttt{classmethod} and \texttt{staticmethod}]{codes/examples/classmethods.py}
	\end{frame}
	
	
	\begin{frame}{\texttt{property} Decorator: Getter and Setter}
		\textbf{What is the \texttt{property} Decorator?}
		\begin{itemize}
			\item \texttt{property} is a built-in decorator in Python used to create managed attributes.
			\item It allows defining a method as a getter and optionally as a setter or deleter.
			\item Enables encapsulation by controlling access to an attribute.
		\end{itemize}
	\end{frame}
	
	
	\begin{frame}[fragile]{Example Code:}
		\lstinputlisting[language=Python, caption=Using \texttt{property} with Getter and Setter]{codes/examples/property-getter-setter.py}
	\end{frame}
	
	
	\begin{frame}{\texttt{type} and \texttt{isinstance} in Python}
		\textbf{What are \texttt{type} and \texttt{isinstance}?}
		\begin{itemize}
			\item \texttt{type}: Returns the type of an object.
			\item \texttt{isinstance}: Checks if an object is an instance of a specific class or a subclass.
		\end{itemize}
		
				
		\textbf{Key Points:}
		\begin{itemize}
			\item \texttt{type(obj1) == type(obj2)} checks if two objects have the exact same type.
			\item \texttt{isinstance} considers inheritance, making it more flexible for class hierarchies.
			\item \texttt{type} is strict and only checks the direct class of an object.
		\end{itemize}


	\end{frame}
	
	\begin{frame}[fragile]{Example: Using \texttt{type}, \texttt{isinstance}}
			\lstinputlisting[language=Python, caption=Type Checking in Python]{codes/examples/type-isinstanceof.py}
	\end{frame}
	
	
	
	\begin{frame}
		\begin{minipage}{.5\linewidth}
				\begin{figure}
				\centering
				\includegraphics[width=0.4\linewidth]{assets/type-loop}
				\caption{}
				\label{fig:type-loop}
			\end{figure}
		\end{minipage}
		\begin{minipage}{.5\linewidth}
			\begin{figure}
				\centering
				\includegraphics[width=0.7\linewidth]{assets/isinstanceof-loop}
				\caption{}
				\label{fig:isinstanceof-loop}
			\end{figure}
		\end{minipage}
	\end{frame}
	

	\begin{frame}[fragile]{Metaclasses in Python}
		\textbf{What is a Metaclass?}
		\begin{itemize}
			\item A metaclass is a class of a class that defines how classes behave.
			\item It controls the creation and behavior of classes, similar to how classes control instances.
			\item By default, Python classes are instances of the \texttt{type} metaclass.
		\end{itemize}
	
		\textbf{Example: Custom Metaclass}
		\begin{lstlisting}[language=Python, caption=Defining and Using a Metaclass]
# Define a custom metaclass
class MyMeta(type):
	def __new__(cls, name, bases, dct):
		print(f"Creating class: {name}")
		dct['greet'] = lambda self: f"Hello from {name}!"
		return super().__new__(cls, name, bases, dct)

# Use the custom metaclass
class MyClass(metaclass=MyMeta):
	pass

# Create an instance and use the dynamically added method
obj = MyClass()
print(obj.greet())  # Output: Hello from MyClass!
		\end{lstlisting}
	\end{frame}
	
	
	
	\begin{frame}{Singleton Design Pattern}
		\textbf{What is the Singleton Pattern?}
		\begin{itemize}
			\item Ensures that a class has only one instance throughout the application.
			\item Provides a single, global access point to that instance.
			\item Often used for managing shared resources like logging, configuration settings, or database connections.
		\end{itemize}
		
		\textbf{Key Features:}
		\begin{itemize}
			\item Controls object creation to guarantee a single instance.
			\item Improves resource management by avoiding duplicate instances.
			\item Simplifies access to shared resources across the application.
		\end{itemize}
		
		\textbf{When to Use It:}
		\begin{itemize}
			\item When exactly one object is required to coordinate actions across the system.
			\item For shared objects that are expensive to create or manage.
			\item To provide global access to a single point of control.
		\end{itemize}
	\end{frame}
	
	
	\begin{frame}[fragile]{Singleton Pattern Using Metaclass}
		\textbf{Example: Singleton Implementation with Metaclass}
		\lstinputlisting[language=Python, caption=Singleton Using Metaclass]{codes/examples/singleton.py}
	\end{frame}
	
	
	\begin{frame}{Logging in Python}
		\textbf{What is Logging?}
		\begin{itemize}
			\item Logging records events, errors, and informational messages during the execution of a program.
			\item Essential for debugging, monitoring, and maintaining production systems.
		\end{itemize}
		
		\textbf{Singleton Architecture in Logging:}
		\begin{itemize}
			\item Python's \texttt{logging} library uses the Singleton pattern.
			\item Ensures a single \texttt{Logger} instance is shared across the application for consistent behavior.
		\end{itemize}
		
		\textbf{Logging Levels:}
		\begin{itemize}
			\item \texttt{DEBUG}: Detailed information, typically for diagnosing problems.
			\item \texttt{INFO}: General information about program execution.
			\item \texttt{WARNING}: Indication of potential issues or unexpected situations.
			\item \texttt{ERROR}: Logs errors that occur during execution but allow the program to continue.
			\item \texttt{CRITICAL}: Severe errors causing the program to abort or require immediate attention.
		\end{itemize}
	\end{frame}
		
	\begin{frame}{Logging in Python}
		\textbf{Usage in Python:}
		\begin{itemize}
			\item \texttt{logger = logging.getLogger(\_\_name\_\_)}:
			\begin{itemize}
				\item Creates or retrieves a logger for the current module.
				\item Used in each module to log events consistently.
			\end{itemize}
			\item In class-based designs:
			\begin{itemize}
				\item Each class can have its specific logger using \texttt{getLogger()}.
			\end{itemize}
		\end{itemize}
		
		\textbf{Key Benefits:}
		\begin{itemize}
			\item Centralized and consistent logging across the application.
			\item Thread-safe and configurable for multiple outputs (e.g., console, files).
		\end{itemize}
	\end{frame}
	
	
	\begin{frame}[fragile]{Example: Using Python Logging}
		\textbf{Python Logging Example with Levels}
		\begin{lstlisting}[language=Python, caption=Basic Logging Example]
import logging

# Configure logging
logging.basicConfig(
	level=logging.DEBUG,
	format='%(asctime)s-%(name)s-%(levelname)s-%(message)s'
)

# Create a logger
logger = logging.getLogger(__name__)

# Logging messages with different levels
logger.debug("This is a DEBUG message.")
logger.info("This is an INFO message.")
logger.warning("This is a WARNING message.")
logger.error("This is an ERROR message.")
logger.critical("This is a CRITICAL message.")
		\end{lstlisting}
	\end{frame}
	
	
	
	\begin{frame}{List Comprehensions in Python}
		\textbf{What are List Comprehensions?}
		\begin{itemize}
			\item A concise way to create lists in Python.
			\item Combines the functionality of loops and \texttt{if} statements in a single line.
			\item Improves readability and reduces boilerplate code.
		\end{itemize}
		
		\textbf{Syntax:}
		\begin{itemize}
			\item \texttt{[expression for item in iterable if condition]}
		\end{itemize}
	
	\end{frame}
	\begin{frame}[fragile]{List Comprehensions in Python}
		\textbf{Examples:}
		\begin{lstlisting}[language=Python, caption=Examples of List Comprehensions]
# Generate a list of squares
squares = [x**2 for x in range(10)]  
# Output: [0, 1, 4, 9, 16, 25, 36, 49, 64, 81]

# Filter even numbers
evens = [x for x in range(10) if x % 2 == 0]
# Output: [0, 2, 4, 6, 8]

# Nested comprehension (flatten a 2D list)
matrix = [[1, 2], [3, 4]]
flattened = [num for row in matrix for num in row]
# Output: [1, 2, 3, 4]

# Conditional expressions
labels = ['Even' if x % 2 == 0 else 'Odd' for x in range(5)]
# Output: ['Even', 'Odd', 'Even', 'Odd', 'Even']
		\end{lstlisting}
		
		\textbf{Key Benefits:}
		\begin{itemize}
			\item Makes code more compact and expressive.
			\item Often faster than equivalent for-loops due to optimization.
			\item Versatile for data transformations and filtering.
		\end{itemize}
	\end{frame}
	
	
	
	\begin{frame}{Generators in Python}
		\textbf{What are Generators?}
		\begin{itemize}
			\item Generators are a concise way to create iterators in Python.
			\item Similar to list comprehensions but use parentheses instead of square brackets.
			\item They generate items lazily, producing one item at a time, which is memory-efficient.
		\end{itemize}
		
		\textbf{Syntax:}
		\begin{itemize}
			\item \texttt{(expression for item in iterable if condition)}
		\end{itemize}
		
	\end{frame}
	\begin{frame}[fragile]{Examples of Generators in Python}
		\begin{lstlisting}[language=Python, caption=Examples of Generators]
# Generator for squares
squares_gen = (x**2 for x in range(10))
print(next(squares_gen))  # Output: 0
print(next(squares_gen))  # Output: 1

# Filter even numbers
evens_gen = (x for x in range(10) if x % 2 == 0)
print(list(evens_gen))  # Output: [0, 2, 4, 6, 8]

# Infinite sequence (using generator function)
def infinite_numbers():
	num = 0
	while True:
		yield num
		num += 1

gen = infinite_numbers()
print(next(gen))  # Output: 0
print(next(gen))  # Output: 1
		\end{lstlisting}

	\end{frame}
	
	
	\begin{frame}{\texttt{Dataset} Class in PyTorch}
		\textbf{What is \texttt{Dataset}?}
		\begin{itemize}
			\item \texttt{torch.utils.data.Dataset} is an abstract class in PyTorch used to represent a dataset.
			\item It provides a standard way to load and preprocess data for training machine learning models.
			\item Custom datasets can be created by subclassing \texttt{Dataset}.
		\end{itemize}
		
		\textbf{Key Methods:}
		\begin{itemize}
			\item \texttt{\_\_len\_\_()}: Returns the size of the dataset.
			\item \texttt{\_\_getitem\_\_()}: Retrieves a single data sample and its label by index.
		\end{itemize}
		
	\end{frame}
	\begin{frame}[fragile]{\texttt{Dataset} Class in PyTorch}
		\textbf{Custom Dataset Example:}
		\begin{lstlisting}[language=Python, caption=Defining a Custom Dataset]
			from torch.utils.data import Dataset
			
class CustomDataset(Dataset):
	def __init__(self, data, labels):
		self.data = data
		self.labels = labels
			
	def __len__(self):
		return len(self.data)
			
	def __getitem__(self, idx):
		return self.data[idx], self.labels[idx]
			
	# Example usage
	data = [[1, 2], [3, 4], [5, 6]]
	labels = [0, 1, 0]
	dataset = CustomDataset(data, labels)
	print(len(dataset))  # Output: 3
	print(dataset[0])    # Output: ([1, 2], 0)
		\end{lstlisting}

	\end{frame}
	
	
	
	
	
	\begin{frame}[allowframebreaks]
		\frametitle{References}
		\printbibliography
	\end{frame}
	
	%----------------------------------------------------------------------------------------
	
\end{document} 

